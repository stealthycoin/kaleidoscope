\documentclass[twocolumn]{paper}
\usepackage{authblk}
\usepackage{listings}
\usepackage{amsmath}
\lstset{language=Haskell}

\title{Usable Functional Reactive Programming with Slipstreams}
\author[*]{Morgan McDermott}
\author[**]{John Carlyle}
\affil[*]{University of flightless dragons}
\affil[**]{University of angry bees}

\begin{document}
\maketitle
\section{Abstract}
We give a definition of \textit{Functional Reactive Programming} (FRP) for use in a javascript based environment that allows for the manipulation of DOM elements and remote servers through the use of sinks and sources in the network. There is also an implementation of the FRP network using a DSL to specify the varying layers of the network, this allows a programmer to easily design networks to accomplish tasks by simply specifying the layers and how the layers interact. Our implementation builds on previous ones by being more practical, the DSL arises naturally from our definition of FRP and the DSL is simple to use and requires no knowledge of the underlying FRP concepts. End sales pitch.

\section{Introduction}
Functional Reactive Programming is the programming paradigm based on the key concepts of \textit{behaviors} and \textit{reactivity}. Behaviors are varying with respect to time and are typically modeled as a stream of values. These values can also be considered events depending on the granularity of the stream. Reactivity is defined as a reaction to an event, when a node recieves a new value it reacts by changing its value and. 

\section{Our contributions}
pretty much none other than an implementation

and begin from awesome schoools

\section{Terminology}
\begin{description}
\item[element] Any value or computation that relys upon previous computations or values. An element is analogous to a node in a graph. The element is connected to various other elements by means of some function. All elements connected to element $e$ can be split into two sets $s_1$ all elements supplying input to $e$ and $s_2$ all elements recieving their input from $e$. $s_1$ is called the predecessor set of $e$ or $s_1 = pre(e)$ and $s_2$ is the successor set of $e$ or $s_2 = suc(e)$

\item[signal] It is useful to think of the values being sent from one element to another as a signal being broadcast continuously. The element recieving this signal, adjusts its own outgoing signal according to a predefined set of rules, or a function.

\item[behavior] Using the previous definition of an element and a signal, an element can be thought of as a function that maps from one signal to another. More concretly\\
  $type~element~a~b = Signal ~a \rightarrow Signal ~b$


\item[network] Simply a set of elements that form a connected component. Multiple networks can be used to describe different unrelated components of a particular interface or problem space.

\item[layer] A layer is a set of elements in a network where the intersection of their mutual union of predecessor and sucessor sets is the empty set. More formally $\bigcup_{e_i \in L}{pre(e_i)} \cap \bigcup_{e_j \in L}{suc(e_j)} = \emptyset$ where $L$ is a layer. Breaking a network into layers is helpful because it helps identify dependencies in the network. It also helps introduce structure into the environment which can help organize a network which may have been difficult to think about. It is not always possible to adhere to the the definition of a layer, or is not practical to do so even if it could be done. For example a feedback loop would be by definition a loop on the network, some way of passing a result back to a predacessor. This technique is useful for maintaining state in a network amongst other uses.



\item[space-time leak] In a functional reactive service any particular element can depend on values that are far in the past. Clearly the longer the program is executing, the longer the history of events, or the longer the stream is. A longer stream takes up more memory and can cause slowdown if the stream is not trimmed after a certain point. Typically a garbage collection service of some kind is used to clean up old events that no longer being depended on. This is analogous to a memory leak in imperative programming. The value is said to be a space-time leak if it is unnessisarly being used for computations when it will have no affect on any current elements.
\end{description}

\begin{thebibliography}{9}
  \bibitem{wormholes}
    Daniel Winograd-Cort, Paul Hudak
    \emph{Wormholes: Introducing Effects to FRP}
    Yale University
    
  \bibitem{fran}
    Conal Elliott and Paul Hudak.
    \emph{Functional Reactive Animation},
    http://conal.net/papers/icfp97\\
    1997


\end{thebibliography}
\end{document}
